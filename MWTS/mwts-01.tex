%%%%%%%%%%%%%%%%%%%%%%% file template.tex %%%%%%%%%%%%%%%%%%%%%%%%%
%
% This is a general template file for the LaTeX package SVJour3
% for Springer journals.          Springer Heidelberg 2006/03/15
%
% Copy it to a new file with a new name and use it as the basis
% for your article. Delete % signs as needed.
%
% This template includes a few options for different layouts and
% content for various journals. Please consult a previous issue of
% your journal as needed.
%
%%%%%%%%%%%%%%%%%%%%%%%%%%%%%%%%%%%%%%%%%%%%%%%%%%%%%%%%%%%%%%%%%%%
%
% First comes an example EPS file -- just ignore it and
% proceed on the \documentclass line
% your LaTeX will extract the file if required
\begin{filecontents*}{example.eps}
%!PS-Adobe-3.0 EPSF-3.0
%%BoundingBox: 19 19 221 221
%%CreationDate: Mon Sep 29 1997
%%Creator: programmed by hand (JK)
%%EndComments
gsave
newpath
  20 20 moveto
  20 220 lineto
  220 220 lineto
  220 20 lineto
closepath
2 setlinewidth
gsave
  .4 setgray fill
grestore
stroke
grestore
\end{filecontents*}
%
\documentclass{article}
%\documentclass{svjour3}                     % onecolumn (standard format)
%\documentclass[smallextended]{svjour3}     % onecolumn (second format)
%\documentclass[twocolumn]{svjour3}         % twocolumn
%
%\smartqed  % flush right qed marks, e.g. at end of proof
%
\usepackage{graphicx}
%
\usepackage{mathptmx}      % use Times fonts if available on your TeX system
%
% insert here the call for the packages your document requires
%\usepackage{latexsym}
% etc.
\usepackage{amsmath, amssymb}
%\usepackage{amsthm}
%
% please place your own definitions here and don't use \def but
\newcommand{\nipar}{\par\noindent\ignorespaces}

%
% Insert the name of "your journal" with
%\journalname{Health Care Management Science}
%

\begin{document}

\title{A multi-week implicit tour scheduling model with applications in healthcare
%\thanks{Grants or other notes
%about the article that should go on the front page should be
%placed here. General acknowledgments should be placed at the end of the article.}
}
%\subtitle{Do you have a subtitle?\\ If so, write it here}

%\titlerunning{Short form of title}        % if too long for running head

\author{
}

%\authorrunning{Short form of author list} % if too long for running head

%\institute{}

%\date{Received: date / Accepted: date}
% The correct dates will be entered by the editor


%\maketitle

%\begin{abstract}


%\keywords{tour scheduling \and healthcare \and optimiization}
% \PACS{PACS code1 \and PACS code2 \and more}
% \subclass{MSC code1 \and MSC code2 \and more}
%\end{abstract}

\section{Introduction}
\label{sec-intro}
Staff scheduling in healthcare continues to be a vexing problem that consumes valuable managerial resources and whose complexity can impact the quality of care delivered and the job satisfaction of its highly skilled labor force. \cite{}

Wide range of staff scheduling problems whose details depend on the specific healthcare delivery subsystem we are talking about. Examples: recovery room nurses, floor nurses, lab technicians, pharmacists and pharmacy technicians, surgical techs, ED nurses and techs, transporters.

Introduce motivation for tactical tour scheduling problems via real projects

For many systems, the kernel of the problem is a tour scheduling problem in which staffing levels needed vary both by time of day and day of week. 

\section{Tour scheduling problems}

\subsection{Scheduling environment}

Features of tour scheduling problems including notions of planning cycle, periods, required staffing levels, shifts, tours, tour types, intra-tour start time flexibility, 

\subsection{Problem size explosion}

\subsection{Implicit modeling of tours}




\section{Related Work}
\label{sec-relatedwork}
Review one week implicit tour scheduling models including my AOR 2004 paper.

Recent multi-week implicit tour scheduling paper

 





\section{Definitions and notation}
\label{sec-mwts}

% Use bold for terms that I define

Let $W$ be the number of weeks in the planning cycle and $P$ be the number of time periods per day. Typical values might be $W=4$ and $P=24$ to model a problem where staffing requirements are specified hourly and shifts can only start on the hour. The total number of periods in the planning cycle is den

\[
\begin{array}{lll}
n_W & = & \mbox{number of weeks in the planning cycle} \\
n_P & = & \mbox{number of periods per day} \\
n_C & = & 7 n_W n_P\mbox{ number of periods in planning cycle} \\
\end{array}
\]

Define a number of sets of indices.

\[
\begin{array}{lll}
P & = & \{1,2,\ldots n_P\} \\
D & = & \{1,2,\ldots 7\} \\
W & = & \{1,2,\ldots n_W\} \\
\end{array}
\]


So, each period in the planning cycle is defined by a tuple $(i,d,w) \in B$ where $B = P \times D \times W$.



%model_phase1.PERIODS = RangeSet(1,model_phase1.n_prds_per_day)
%model_phase1.WINDOWS = RangeSet(1,model_phase1.n_prds_per_day)
%model_phase1.DAYS = RangeSet(1,model_phase1.n_days_per_week)
%model_phase1.WEEKS = RangeSet(1,model_phase1.n_weeks)


%#-- Shift Lengths
%model_phase1.n_lengths = Param(within=PositiveIntegers)    # Number of shift lengths
%model_phase1.LENGTHS = RangeSet(1,model_phase1.n_lengths)
%model_phase1.lengths = Param(model_phase1.LENGTHS)  # Vector of shift lengths

To smooth the postpartum occupancy across days of the week, we introduce two, non-negative, dummy variables, $M^{U}$ and $M^{L}$. These are used to bound the mean occupancy across the week from above and below -- see Equation \eqref{eqn:sandwich}. Then we minimize the difference between the bounds.

\newpage
\nipar {\bf Model:} OB-SMOOTH
\nipar
\nipar Minimize
\begin{gather}
M^{U} - M^{L} \label{eqn:obj} 
\end{gather}
Subject to
\begin{gather}
0 \leq M^{L} \leq \mu_{i,j}^{4} \leq M^{U} \quad \text{($i=1,2,\ldots P$, $j=1,2,\ldots 7$)} \label{eqn:sandwich} \\ 
L_{i,j}^{a} \leq \Lambda_{i,j}^{a} \leq U_{i,j}^{a} \quad \text{($a=5,6, i=1,2,\ldots P$, $j=1,2,\ldots 7$)} \label{eqn:schedbin} \\
L_{.,j}^{a} \leq \sum_{i=1}^{P}\Lambda_{i,j}^{a} \leq U_{.,j}^{a} \quad \text{($a=5,6$, $j=1,2,\ldots 7$)} \label{eqn:schedday} \\
L^{a} \leq \sum_{i=1}^{P}\sum_{j=1}^{7}\Lambda_{i,j}^{a} \leq U^{a} \quad \text{($a=5,6$)} \label{eqn:schedweek} \\ 
\text{Equations \eqref{eqn:arrrate1}-\eqref{eqn:arrrate11} for arrival rates }  \\
\text{Equation \eqref{eqn:discharge} for discharge rates}  \\
\text{Equation \eqref{eqn:flowcons} for conservation of flow}  \\
\text{Equation \eqref{eqn:meanocc} for mean occupancy} \\
\text{Equations \eqref{eqn:varocc2}-\eqref{eqn:varocc1} for variance of occupancy }  \\
\Lambda_{i,j}^{a} \geq 0 \text{ and integer}, \quad \text{($a=5,6, i=1,2,\ldots P$, $j=1,2,\ldots 7)$} \label{eqn:schedint}
\end{gather} 

Equation \eqref{eqn:schedbin} sets upper and lower bounds on the number of scheduled inductions and scheduled cesareans by time period of day and day of week. Equations \eqref{eqn:schedday} and \eqref{eqn:schedweek} set upper and lower bounds on the number of scheduled inductions and scheduled cesareans by day of week and for the entire week, respectively. The remaining constraints constitute the patient flow portion of the model. Note that the only decision variables are $\Lambda_{i,j}^{5}$, the number of scheduled inductions, and $\Lambda_{i,j}^{6}$, the number of scheduled cesareans. The model OB-SMOOTH is a standard mixed integer linear program. 


\subsection{Computational experiments}
\label{sec-computational}

Solve realistic problems. No competitive models to which to compare to other than reporting the size of comparable explicit tour scheduling models. I guess I could compare to the 5/7 model for the restricted environment for which it was designed.

\section{Just some templates to use}


\begin{table}
  \centering
  \caption{Patient care units}\label{table:units}
\begin{tabular}{cll}\hline
 Unit \#       & Unit Name    & Abbr.   \\ \hline
  1       & Labor \& Delivery    & LD    \\
  2       & Recovery room   & R    \\
  3       & Cesarean section procedure area    & C    \\
  4       & Postpartum unit    & PP   \\ \hline  
\end{tabular}
\end{table}



\begin{equation}
\label{eqn:flowcons}
\lambda_{i,j}^{h,s(h,m)}=D_{i,j}^{h,r(h,m)} \quad \text{($i=1,2,\ldots P$, $j=1,2,\ldots 7$, $h \in {\cal H}$, $m=2,3\ldots n_{h}$).}
\end{equation}
\section{Conclusions and Future Work}
\label{sec-conclusions}


BLAH, BLAH






%\begin{acknowledgements}
%If you'd like to thank anyone, place your comments here
%and remove the percent signs.
%\end{acknowledgements}


% BibTeX users please use one of
% BibTeX users please use one of
%\bibliographystyle{spbasic}      % basic style, author-year citations
%\bibliographystyle{spmpsci}      % mathematics and physical sciences
%\bibliographystyle{spphys}       % APS-like style for physics
%\bibliography{obsched-paper}   % name your BibTeX data base
%\bibliographystyle{unsrt}


\end{document}
% end of file template.tex

