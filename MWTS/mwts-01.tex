%%%%%%%%%%%%%%%%%%%%%%% file template.tex %%%%%%%%%%%%%%%%%%%%%%%%%
%
% This is a general template file for the LaTeX package SVJour3
% for Springer journals.          Springer Heidelberg 2006/03/15
%
% Copy it to a new file with a new name and use it as the basis
% for your article. Delete % signs as needed.
%
% This template includes a few options for different layouts and
% content for various journals. Please consult a previous issue of
% your journal as needed.
%
%%%%%%%%%%%%%%%%%%%%%%%%%%%%%%%%%%%%%%%%%%%%%%%%%%%%%%%%%%%%%%%%%%%
%

%
\documentclass{article}
%\documentclass{svjour3}                     % onecolumn (standard format)
%\documentclass[smallextended]{svjour3}     % onecolumn (second format)
%\documentclass[twocolumn]{svjour3}         % twocolumn
%
%\smartqed  % flush right qed marks, e.g. at end of proof
%
\usepackage{graphicx}
%
\usepackage{mathptmx}      % use Times fonts if available on your TeX system
%
% insert here the call for the packages your document requires
%\usepackage{latexsym}
% etc.
\usepackage[fleqn]{amsmath}
\usepackage{amssymb}
\usepackage{amsfonts}
\usepackage{euscript}
%\usepackage{amsthm}
%
% please place your own definitions here and don't use \def but
\newcommand{\nipar}{\par\noindent\ignorespaces}

%
% Insert the name of "your journal" with
%\journalname{Health Care Management Science}
%

\begin{document}

\title{A multi-week implicit tour scheduling model with applications in healthcare
%\thanks{Grants or other notes
%about the article that should go on the front page should be
%placed here. General acknowledgments should be placed at the end of the article.}
}
%\subtitle{Do you have a subtitle?\\ If so, write it here}

%\titlerunning{Short form of title}        % if too long for running head

\author{
}

%\authorrunning{Short form of author list} % if too long for running head

%\institute{}

%\date{Received: date / Accepted: date}
% The correct dates will be entered by the editor


%\maketitle

%\begin{abstract}


%\keywords{tour scheduling \and healthcare \and optimiization}
% \PACS{PACS code1 \and PACS code2 \and more}
% \subclass{MSC code1 \and MSC code2 \and more}
%\end{abstract}

\section{Introduction}
\label{sec-intro}
Staff scheduling in healthcare continues to be a vexing problem that consumes valuable managerial resources and whose complexity can impact the quality of care delivered and the job satisfaction of its highly skilled labor force \cite{}. Complicating matters is the fact that there are a wide range of staff scheduling problems whose details depend on the specific healthcare delivery subsystem being considered as well as on the decision making and temporal context of the problem at hand. For example, the ongoing task of creating four week schedules for a pool of nurses working in a set of nursing units is quite different than trying to determine if a mix of eight and ten-hour shifts provides a better match of staffing levels to highly variable demand in an emergency department or post-anaesthesia care unit. The former problem is an example of \textit{operational staff scheduling} while the latter is better described as \textit{tactical staff scheduling} \cite{isken:aor2004}.

From a modeling perspective, in an operational staff scheduling model there would be decision variables representing specific employees whereas in a tactical model it would be more common to find decision variables representing the number of employees working different types of schedules.

Commercial staff scheduling software state of art overview. Many systems are essentially shift based ``schedulators''. While schedulators are useful, they avoid the toughest part of the problem. Imagine using Google Maps for finding the shortest driving route between two locations and having to manually input the route so that Google Maps can tell you the length of your route, estimated arrival time and other route attributes.

Should we mention any specific products such as ANSOS?

Examples: recovery room nurses, floor nurses, lab technicians, pharmacists and pharmacy technicians, surgical techs, ED nurses and techs, transporters.

Introduce motivation for tactical tour scheduling problems via real projects. Value of being able to produce realistic multi-week schedules to illustrate how new scheduling practices might actually work.

For many systems, the kernel of the problem is a tour scheduling problem in which staffing levels needed vary both by time of day and day of week. 

Introduce project website containing Pyomo implementation of this model, motivation for doing this, relationship to OBSched

\section{Tour scheduling problems}

\subsection{Scheduling environment}
\label{sec-schedenv}

Features of tour scheduling problems including notions of planning cycle, periods, required staffing levels, shifts, tours, tour types, 

In \cite{isken:2004}, intra-tour start time flexibility was implicitly modeled using start time windows and associated control constraints. That same approach can be used in this multi-week model. However, it significantly complicates notation to include start time windows and since the focus of this paper is on multi-week modeling, we will restrict our attention to the case of no intra-tour start time flexibility. The Pyomo implementation available at the project website includes intra-tour start time flexibility.  Similarly, for purposes of this paper, we only consider the case where weekends are defined as Saturday and Sunday. In reality, depending on the start time of the shift, it may be more appropriate to consider Friday and Saturday as the weekend. Again, the Pyomo implementations includes this practical complication. 

To explicitly represent a tour we need of list of days worked, including the shift start times and lengths over the planning cycle. For example, consider the following very simple scenario for a two week planning cycle:

\begin{itemize}
\item only allowable tour type calls for five eight-hour shifts per week, M-F,
\item within each tour, all shifts must start at 7a or they must all start at 3p,
\item no restrictions on the number of weekend days worked over the planning cycle.
\end{itemize}

In this case, there are only two possible tours:

\begin{tabular}{|l|c|c|c|c|c|c|c|c|c|c|c|c|c|c|}
\hline 
• & \multicolumn{7}{c|}{Week 1} & \multicolumn{7}{c|}{Week 2} \\ 
\hline 
Tour Variable & Su & M & T & W & U & F & Sa & Su & M & T & W & U & F & Sa \\ 
\hline 
x_1 & x & 7a & 7a & 7a & 7a & 7a & x & x & 7a & 7a & 7a & 7a & 7a & x \\ 
\hline 
x_2 & x & 3p & 3p & 3p & 3p & 3p & x & x & 3p & 3p & 3p & 3p & 3p & x \\ 
\hline 
\end{tabular} 

In this case, the solution to the tour scheduling problem is fully specified by the number of people working Tour 1 and the number working Tour 2 and we only need two tour variables ($x_1$ and $x_2$) in our model. Of course, the number of tour variables can become enormous as we increase our scheduling options such as additional tour types, shift start times and shift lengths. We will explore this issue in the next section. Here, we want to show that there are other ways we could model this very simple scenario in terms of the variables we choose to use. For example, instead of explicit tour variables, we could use one variable to represent the number of people working a specific days worked pattern and another to represent the number of people working each shift. This is called an $implicit modeling$ approach \cite{}.

\begin{tabular}{|l|c|c|c|c|c|c|c|c|c|c|c|c|c|c|}
\hline 
• & \multicolumn{7}{c|}{Week 1} & \multicolumn{7}{c|}{Week 2} \\  
Days Worked Variable & Su & M & T & W & U & F & Sa & Su & M & T & W & U & F & Sa \\ 
\hline 
y_1 & x & On & On & On & On & On & x & x & On & On & On & On & On & x \\  
\hline 
\end{tabular}

\begin{tabular}{|c|c|}
\hline 
Shift Variable & Shift Start Time \\ 
\hline 
s_1 & 7a \\ 
\hline 
s_2 & 3p \\ 
\hline 
\end{tabular}

To specify a valid solution to our tour scheduling problem, we need values for $s_1, s_2$ and $y_1$. For this toy problem, it is easy to see the correspondence between variables in the two approaches. For given values of $x_1$ and $x_2$, $y_1 = x_1 + x_2$, $s_1 = x_1$, and $s_2 = x_2$. While the explicit approach only used two variables, the implicit approach required three variables. However, as we start to model more realistic problems, it will become clear that the implicit approach requires far fewer variables than its explicit counterpart. This reduction will come at a cost in terms of additional constraints as well as in the need for a tour construction post-processing phase. In terms of additional constraints, note that in the implicit approach we must ensure that $y_1 = s_1 + s_2$ if we are going to be able to translate a solution to the implicit problem to an explicit set of tours.

\subsection{Implicit modeling of tours}

Implicit modeling for tour scheduling problems is not new. 

Review one week implicit tour scheduling models including my AOR 2004 paper.

Recent multi-week implicit tour scheduling paper

Also review implicit modeling of breaks and other stuff

\subsection{Problem size explosion}
\label{subsec-explosion}




\section{Related Work}
\label{sec-relatedwork}

If I haven't already covered it above
 

\section{Overview of an Implicit Multi-Week Model}
\label{sec-overview}

As depressingly illustrated in Section \ref{subsec-explosion}, multi-week tour scheduling problems can get massively large. Since even the number of explicit multi-week days worked patterns can get large as the number of weeks increases, our approach is to model these days worked patterns implicitly by focusing on only explicitly modeling the patterns of weekend days worked. This is driven by the observation that it is often the weekends that must be treated in some special way due to the general undesirability of working on weekends as well as the often different required staffing levels seen on weekends. While drastically reducing the total number of variables required, we need to introduce several additional types of variables as well as many control constraints to ensure that a solution to an instance of our implicit model corresponds with a feasible and optimal solution to the equivalent tour scheduling model.



\section{Definitions and notation}
\label{sec-mwts}
%http://en.wikibooks.org/wiki/LaTeX/Mathematics
% Use bold for terms that I define


Planning cycle related terms...

\subsubsection*{Planning cycle parameters}

\begin{flalign*}
n_E & =  \text{number of weeks in the planning cycle} \\
n_P & =  \text{number of periods per day} \\
n_C & =  7 n_W n_P\mbox{ number of periods in planning cycle} \\
\end{flalign*}

\nipar Define a number of sets of indices.

\begin{flalign*}
\mathbb{P} & =  \{1,2,\ldots n_P\} \\
\mathbb{D} & =  \{1,2,\ldots 7\} \\
\mathbb{E} & =  \{1,2,\ldots n_E\} \\
\end{flalign*}

\nipar So, each period in the planning cycle is defined by a tuple $(i,d,w) \in \mathbb{B}$ where $\mathbb{B} = \mathbb{P} \times \mathbb{D} \times \mathbb{W}$.

\subsubsection*{Staffing requirements parameters}

We model the staffing requirements in each period with two sets of variables; one set for minimum staffing requirements that cannot be violated and the other for target staffing requirements that can be violated but where violations are penalized according to an understaffing cost function.

\begin{flalign*}
s_{ijw} = & \text{ target staffing level in period $i$ of day $j$ in week $w$,} \\
m_{ijw} = & \text{ minimum allowable staffing level in period $i$ of day $j$ in week $w$,} \\
&  \text{ for $i \in \mathbb{P},j \in \mathbb{D},w \in \mathbb{E}$.}  
\end{flalign*}

\begin{verbatim}
# Target and minimum staffing levels - this is week specific. We can always allow user to input
# a single week and then repeat it for the other weeks.

model_phase1.dmd_staff = Param(model_phase1.PERIODS,model_phase1.DAYS,model_phase1.WEEKS)
model_phase1.min_staff = Param(model_phase1.PERIODS,model_phase1.DAYS,model_phase1.WEEKS) 
\end{verbatim}

\subsubsection*{Shift length and tour type parameters}
%#-- Shift Lengths
%model_phase1.n_lengths = Param(within=PositiveIntegers)    # Number of shift lengths
%model_phase1.LENGTHS = RangeSet(1,model_phase1.n_lengths)
%model_phase1.lengths = Param(model_phase1.LENGTHS)  # Vector of shift lengths

\begin{flalign*}
n_K& = \text{number of different shift lengths} \\
\mathbb{K}& = \{1,2,\ldots n_K\} \\
l_k& = \text{$k$'th shift length in periods, for $k \in \mathbb{K}$}
\end{flalign*}




%#-- Tour Types
%model_phase1.n_tts = Param(within=PositiveIntegers)  # Number of different tour types
%model_phase1.TTYPES = RangeSet(1,model_phase1.n_tts)
%model_phase1.tt_length_x = Set(model_phase1.TTYPES,ordered=True,)  # Set of allowable length indices by tour type

\begin{flalign*}
n_T& = \text{number of different tour types} \\
\mathbb{T}& = \{1,2,\ldots n_T\} \\
L(t)& = \text{set of shift length indices allowed for tour type $t$, for $t \in \mathbb{T}$}
\end{flalign*}

A_{ijkt}=\begin{cases}
1& \text{1 if a shift of length $k$ starting in period $i$ of day $j$ for tour type $t$ is allowed},\\
& \text{for $i \in \mathbb{P}$, $j \in \mathbb{D}$, $k \in \mathbb{K}$, $t \in \mathbb{T}$.} \\
0& \text{otherwise}.
\end{cases}

\begin{flalign*}
\mathbb{A} & =  \{(i,j,k,t) | A_{ijkt}=1 \} \\
\end{flalign*}

is the set of \textit{allowable shift start times}.

\subsubsection*{Shift and tour type variables}

Instead of explicitly representing tours with one variable per tour, they are implicitly represented using a number of building block variables whose values are coordinated by numerous control constraints to ensure that valid tours can be constructed from these variables.

\textit{Shift variables} represent a single shift with a start day and time, shift length, in a specific week and corresponding to a certain tour type.

\begin{flalign*}
x_{ijkwt} = & \text{ number of shifts of length $k$ starting in period $i$ of day $j$ in week $w$ for} \\
&  \text{ tour type $t$, for $(i,j,k,t) \in \mathbb{A},t \in \mathbb{T}$.}  
\end{flalign*}

\textit{Tour type variables} represent the number of people assigned to each tour type in each start time period. 

\begin{flalign*}
T_{it} = & \text{ number of tours of tour type $t$ assigned to start time period $i$} \\
&  \text{for $i \in \mathbb{P},t \in \mathbb{T}$.}  
\end{flalign*}

\textit{Daily tour type variables} represent the number of people assigned to each tour type in each start time period and working on a given day in a given week. Note that these variables are not shift length specific.

\begin{flalign*}
D_{itjw} = & \text{ number of tours of tour type $t$ assigned to start time period $i$ and working} \\
&  \text{ day $j$ in week $w$, for $i \in \mathbb{P},j \in \mathbb{D}, t \in \mathbb{T}, w \in \mathbb{E}$.} 
\end{flalign*}

\textit{Daily shift tour type variables} represent the number of people assigned to each tour type in each start time period and working a shift of a given length on a given day in a given week. 

\begin{flalign*}
S_{itkjw} = & \text{ number of tours of tour type $t$ assigned to start time period $i$ and working} \\
&  \text{ shift length $k$ on day $j$ in week $w$, for $i \in \mathbb{P},t \in \mathbb{T},k \in \mathbb{K},j \in \mathbb{D}, w \in \mathbb{E}$.} 
\end{flalign*}

Discussion of weekend days worked patterns

\begin{flalign*}
\mathbb{X}(it)& = \text{set of allowable weekend patterns for tour type $i$, for $i \in \mathbb{T}$}
\end{flalign*}

\begin{flalign*}
WW_{itp} = & \text{ number of people working tours of tour type $t$ assigned to start time period $i$} \\
&  \text{ and weekend pattern $p$, for $i \in \mathbb{P},t \in \mathbb{T},p in \mathbb{X}(i,t)$.}  
\end{flalign*}




%TODO Coverage constraints
%TODO Min staff constraints

%TODO Understaffing variables
%TODO Understaffing bounds
%TODO Understaffing cost parameters

%TODO Part time fraction parameter
%TODO Part time fraction constraint

%TODO Budget parameter
%TODO Budget constraint

%TODO Obective function

%TODO DTT_TT_UB_wkendadj constraint - special constraint for tour type working 2 days/week














% Constraints in:
%	TT upper and lower bounds
%   Weekends worked total - coordinates W and T for each (i,t)
%   Weekends and daily tour type
%   Daily tour type and tour type
%   Weekly lower and upper bounds on daily tour type




\newpage
\nipar {\bf Model:} MWTS
\nipar
\nipar Minimize
\begin{gather}
M^{U} - M^{L} \label{eqn:obj} 
\end{gather}
Subject to
\begin{gather}
% Bounds on T
T^{L}_{t} \leq \sum_{i \in \mathbb{P}} T_{it} \leq T^{U}_{t} \quad \text{($t \in \mathbb{T}$)} \label{eqn:tt_ub} \\
%
% Integrate W and T overall
\sum_{x \in \mathbb{X}_{it}}W_{xit} = T_{it} \quad \text{($i \in \mathbb{P}, t \in \mathbb{T}$)} \label{eqn:W_T_tot} \\
%
% Integrate W and DT
\sum_{x \in \mathbb{X}_{it}}A_{xjwt}W_{xit} = D_{itjw} \quad \text{($j \in \{1,7\},w \in \mathbb{E},i \in \mathbb{P}, t \in \mathbb{T}$)} \label{eqn:WW_DT} \\
%
% Integrate DT and T - no more than T can work on any given day
D_{itjw} \leq T{it} \quad \text{($i \in \mathbb{P},t \in \mathbb{T},j \in \mathbb{D},w \in \mathbb{E}$)} \label{eqn:D_T_dailyub} \\
%
%Coordinate DailyShiftWorked and DailyTourType variables for each day of week in each week
\sum_{k \in L(t)}S_{itkjw} = D_{itjw} \quad \text{($i \in \mathbb{P},t \in \mathbb{T},j \in \mathbb{D},w \in \mathbb{E}$)} \label{eqn:S_D_dailyeq} 
\end{gather} 

\begin{gather}
% ----------------- DT and T, days--------------------------------------------------
% Weekly lower and upper bounds on daily tour type
\sum_{d \in \mathbb{D}}D_{itdw} \geq \lambda(t,w)T_{it} \quad \text{($i \in \mathbb{P},t \in \mathbb{T},w \in \mathbb{E}$)} \label{eqn:TT_T_weeklylb} \\
\sum_{d \in \mathbb{D}}D_{itdw} \leq \mu(t,w)T_{it} \quad \text{($i \in \mathbb{P},t \in \mathbb{T},w \in \mathbb{E}$)} \label{eqn:TT_T_weeklyub} \\
%
% Weekly cumulative lower and upper bounds on daily tour type
\sum_{d \in \mathbb{D}} \sum_{z=1}^{w} D_{itdw} \geq \Lambda (t,w)T_{it} \quad \text{($i \in \mathbb{P},t \in \mathbb{T},w \in \mathbb{E}$)} \label{eqn:TT_T_cum_weeklylb} \\
\sum_{d \in \mathbb{D}} \sum_{z=1}^{w} D_{itdw} \leq M(t,w)T_{it} \quad \text{($i \in \mathbb{P},t \in \mathbb{T},w \in \mathbb{E}$)} \label{eqn:TT_T_cum_weeklyub} 
%
\end{gather} 

\begin{gather}
% ----------------- ST and T, days--------------------------------------------------
% Weekly lower and upper bounds on shift tour type based on days worked
\sum_{d \in \mathbb{D}}\sum_{k \in L(t)}S_{itkjw} \geq \lambda(t,w)T_{it} \quad \text{($i \in \mathbb{P},t \in \mathbb{T},w \in \mathbb{E}$)} \label{eqn:S_T_weeklylb} \\
\sum_{d \in \mathbb{D}}\sum_{k \in L(t)}S_{itkjw} \leq \mu(t,w)T_{it} \quad \text{($i \in \mathbb{P},t \in \mathbb{T},w \in \mathbb{E}$)} \label{eqn:S_T_weeklyub} \\
%
% Weekly cumulative lower and upper bounds on shift tour type based on days worked
\sum_{d \in \mathbb{D}} \sum_{k \in L(t)} \sum_{z=1}^{w} S_{itkjw} \geq \Lambda (t,w)T_{it} \quad \text{($i \in \mathbb{P},t \in \mathbb{T},w \in \mathbb{E}$)} \label{eqn:S_T_cum_weeklylb} \\
\sum_{d \in \mathbb{D}} \sum_{k \in L(t)} \sum_{z=1}^{w} S_{itkjw} \leq M(t,w)T_{it} \quad \text{($i \in \mathbb{P},t \in \mathbb{T},w \in \mathbb{E}$)} \label{eqn:S_T_cum_weeklyub} 
%
\end{gather} 

\begin{gather}
% ----------------- ST and T, days, shiftlen--------------------------------------------------
% Weekly lower and upper bounds on shift tour type by shiftlen based on days worked
\sum_{d \in \mathbb{D}}S_{itkjw} \geq \lambda_{s}(t,w)T_{it} \quad \text{($i \in \mathbb{P},t \in \mathbb{T},k \in L(t),w \in \mathbb{E}$)} \label{eqn:S_T_shiftlen_weeklylb} \\
\sum_{d \in \mathbb{D}}S_{itkjw} \leq \mu_{s}(t,w)T_{it} \quad \text{($i \in \mathbb{P},t \in \mathbb{T},k \in L(t),w \in \mathbb{E}$)} \label{eqn:S_T_shiftlen_weeklyub} \\
%
% Weekly cumulative lower and upper bounds on shift tour type by shiftlen based on days worked
\sum_{d \in \mathbb{D}} \sum_{z=1}^{w} S_{itkjw} \geq \Lambda_{s}(t,w)T_{it} \quad \text{($i \in \mathbb{P},t \in \mathbb{T},k \in L(t),w \in \mathbb{E}$)} \label{eqn:S_T_cum_shiftlen_weeklylb} \\
\sum_{d \in \mathbb{D}} \sum_{z=1}^{w} S_{itkjw} \leq M_{s}(t,w)T_{it} \quad \text{($i \in \mathbb{P},t \in \mathbb{T},k \in L(t),w \in \mathbb{E}$)} \label{eqn:S_T_cum_shiftlen_weeklyub} 
%
\end{gather} 

\begin{gather}
% ----------------- ST and T, periods --------------------------------------------------
% Weekly lower and upper bounds on shift tour type based on periods worked
\sum_{d \in \mathbb{D}}\sum_{k \in L(t)}l_k S_{itkjw} \geq \psi(t,w)T_{it} \quad \text{($i \in \mathbb{P},t \in \mathbb{T},w \in \mathbb{E}$)} \label{eqn:S_T_weeklylb_periods} \\
\sum_{d \in \mathbb{D}}\sum_{k \in L(t)}l_k S_{itkjw} \leq \omega(t,w)T_{it} \quad \text{($i \in \mathbb{P},t \in \mathbb{T},w \in \mathbb{E}$)} \label{eqn:S_T_weeklyub_periods} \\
%
% Weekly cumulative lower and upper bounds on shift tour type based on periods worked
\sum_{d \in \mathbb{D}}\sum_{k \in L(t)}\sum_{z=1}^{w}l_k S_{itkjw} \geq \Psi(t,w)T_{it} \quad \text{($i \in \mathbb{P},t \in \mathbb{T},w \in \mathbb{E}$)} \label{eqn:S_T_cum_weeklylb_periods} \\
\sum_{d \in \mathbb{D}}\sum_{k \in L(t)}\sum_{z=1}^{w}l_k S_{itkjw} \leq \Omega(t,w)T_{it} \quad \text{($i \in \mathbb{P},t \in \mathbb{T},w \in \mathbb{E}$)} \label{eqn:S_T_cum_weeklyub_periods} 
%
\end{gather} 

\begin{gather}
% ----------------- ST and T, periods, shiftlen --------------------------------------------------
% Weekly lower and upper bounds on shift tour type based on periods worked
\sum_{d \in \mathbb{D}}\sum_{k \in L(t)}l_k S_{itkjw} \geq \psi_s(t,w)T_{it} \quad \text{($i \in \mathbb{P},t \in \mathbb{T},w \in \mathbb{E}$)} \label{eqn:S_T_weeklylb_periods} \\
\sum_{d \in \mathbb{D}}\sum_{k \in L(t)}l_k S_{itkjw} \leq \omega_s(t,w)T_{it} \quad \text{($i \in \mathbb{P},t \in \mathbb{T},w \in \mathbb{E}$)} \label{eqn:S_T_weeklyub_periods} \\
%
% Weekly cumulative lower and upper bounds on shift tour type based on periods worked
\sum_{d \in \mathbb{D}}\sum_{z=1}^{w}l_k S_{itkjw} \geq \Psi_s(t,w)T_{it} \quad \text{($i \in \mathbb{P},t \in \mathbb{T},k \in L(t), w \in \mathbb{E}$)} \label{eqn:S_T_cum_weeklylb_periods} \\
\sum_{d \in \mathbb{D}}\sum_{z=1}^{w}l_k S_{itkjw} \leq \Omega_s(t,w)T_{it} \quad \text{($i \in \mathbb{P},t \in \mathbb{T},k \in L(t),w \in \mathbb{E}$)} \label{eqn:S_T_cum_weeklyub_periods} 
% Non-negativity and integer
T_{it} \geq 0 \text{ and integer}, \quad \text{($i \in \mathbb{P}, t \in \mathbb{T}$)} \label{eqn:TT_int}
\end{gather} 






\subsection{Computational experiments}
\label{sec-computational}

Solve realistic problems. No competitive models to which to compare to other than reporting the size of comparable explicit tour scheduling models. I guess I could compare to the 5/7 model for the restricted environment for which it was designed.

\section{Just some templates to use}


\begin{table}
  \centering
  \caption{Patient care units}\label{table:units}
\begin{tabular}{cll}\hline
 Unit \#       & Unit Name    & Abbr.   \\ \hline
  1       & Labor \& Delivery    & LD    \\
  2       & Recovery room   & R    \\
  3       & Cesarean section procedure area    & C    \\
  4       & Postpartum unit    & PP   \\ \hline  
\end{tabular}
\end{table}



\begin{equation}
\label{eqn:flowcons}
\lambda_{i,j}^{h,s(h,m)}=D_{i,j}^{h,r(h,m)} \quad \text{($i=1,2,\ldots P$, $j=1,2,\ldots 7$, $h \in {\cal H}$, $m=2,3\ldots n_{h}$).}
\end{equation}
\section{Conclusions and Future Work}
\label{sec-conclusions}


BLAH, BLAH






%\begin{acknowledgements}
%If you'd like to thank anyone, place your comments here
%and remove the percent signs.
%\end{acknowledgements}


% BibTeX users please use one of
% BibTeX users please use one of
%\bibliographystyle{spbasic}      % basic style, author-year citations
%\bibliographystyle{spmpsci}      % mathematics and physical sciences
%\bibliographystyle{spphys}       % APS-like style for physics
%\bibliography{obsched-paper}   % name your BibTeX data base
%\bibliographystyle{unsrt}


\end{document}
% end of file template.tex

