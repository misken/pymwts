%%%%%%%%%%%%%%%%%%%%%%% file template.tex %%%%%%%%%%%%%%%%%%%%%%%%%
%
% This is a general template file for the LaTeX package SVJour3
% for Springer journals.          Springer Heidelberg 2006/03/15
%
% Copy it to a new file with a new name and use it as the basis
% for your article. Delete % signs as needed.
%
% This template includes a few options for different layouts and
% content for various journals. Please consult a previous issue of
% your journal as needed.
%
%%%%%%%%%%%%%%%%%%%%%%%%%%%%%%%%%%%%%%%%%%%%%%%%%%%%%%%%%%%%%%%%%%%
%

%
\documentclass{article}
%\documentclass{svjour3}                     % onecolumn (standard format)
%\documentclass[smallextended]{svjour3}     % onecolumn (second format)
%\documentclass[twocolumn]{svjour3}         % twocolumn
%
%\smartqed  % flush right qed marks, e.g. at end of proof
%
\usepackage{graphicx}
%
\usepackage{mathptmx}      % use Times fonts if available on your TeX system
%
% insert here the call for the packages your document requires
%\usepackage{latexsym}
% etc.
\usepackage[fleqn]{amsmath}
\usepackage{amssymb}
\usepackage{amsfonts}
\usepackage{euscript}
%\usepackage{amsthm}
%
% please place your own definitions here and don't use \def but
\newcommand{\nipar}{\par\noindent\ignorespaces}

%
% Insert the name of "your journal" with
%\journalname{Health Care Management Science}
%

\begin{document}

\title{A multi-week implicit tour scheduling model with applications in healthcare
%\thanks{Grants or other notes
%about the article that should go on the front page should be
%placed here. General acknowledgments should be placed at the end of the article.}
}
%\subtitle{Do you have a subtitle?\\ If so, write it here}

%\titlerunning{Short form of title}        % if too long for running head

\author{
}

%\authorrunning{Short form of author list} % if too long for running head

%\institute{}

%\date{Received: date / Accepted: date}
% The correct dates will be entered by the editor


%\maketitle

%\begin{abstract}


%\keywords{tour scheduling \and healthcare \and optimiization}
% \PACS{PACS code1 \and PACS code2 \and more}
% \subclass{MSC code1 \and MSC code2 \and more}
%\end{abstract}

\section{Introduction}
\label{sec-intro}
Staff scheduling in healthcare continues to be a vexing problem that consumes valuable managerial resources and whose complexity can impact the quality of care delivered and the job satisfaction of its highly skilled labor force. \cite{}

Wide range of staff scheduling problems whose details depend on the specific healthcare delivery subsystem we are talking about. Examples: recovery room nurses, floor nurses, lab technicians, pharmacists and pharmacy technicians, surgical techs, ED nurses and techs, transporters.

Introduce motivation for tactical tour scheduling problems via real projects

For many systems, the kernel of the problem is a tour scheduling problem in which staffing levels needed vary both by time of day and day of week. 

\section{Tour scheduling problems}

\subsection{Scheduling environment}

Features of tour scheduling problems including notions of planning cycle, periods, required staffing levels, shifts, tours, tour types, intra-tour start time flexibility, 

To explicitly represent a tour we need of list of days worked, including the shift start times and lengths over the planning cycle. For example, consider the following very simple scenario for a two week planning cycle:

\begin{itemize}
\item only allowable tour type calls for five eight-hour shifts per week, M-F,
\item within each tour, all shifts must start at 7a or they must all start at 3p,
\item no restrictions on the number of weekend days worked over the planning cycle.
\end{itemize}

In this case, there are only two possible tours:

\begin{tabular}{|l|c|c|c|c|c|c|c|c|c|c|c|c|c|c|}
\hline 
• & \multicolumn{7}{c|}{Week 1} & \multicolumn{7}{c|}{Week 2} \\ 
\hline 
Tour Variable & Su & M & T & W & U & F & Sa & Su & M & T & W & U & F & Sa \\ 
\hline 
x_1 & x & 7a & 7a & 7a & 7a & 7a & x & x & 7a & 7a & 7a & 7a & 7a & x \\ 
\hline 
x_2 & x & 3p & 3p & 3p & 3p & 3p & x & x & 3p & 3p & 3p & 3p & 3p & x \\ 
\hline 
\end{tabular} 

In this case, the solution to the tour scheduling problem is fully specified by the number of people working Tour 1 and the number working Tour 2 and we only need two tour variables ($x_1$ and $x_2$) in our model. Of course, the number of tour variables can become enormous as we increase our scheduling options such as additional tour types, shift start times and shift lengths. We will explore this issue in the next section. Here, we want to show that there are other ways we could model this very simple scenario in terms of the variables we choose to use. For example, instead of explicit tour variables, we could use one variable to represent the number of people working a specific days worked pattern and another to represent the number of people working each shift. This is called an $implicit modeling$ approach \cite{}.

\begin{tabular}{|l|c|c|c|c|c|c|c|c|c|c|c|c|c|c|}
\hline 
• & \multicolumn{7}{c|}{Week 1} & \multicolumn{7}{c|}{Week 2} \\  
Days Worked Variable & Su & M & T & W & U & F & Sa & Su & M & T & W & U & F & Sa \\ 
\hline 
y_1 & x & On & On & On & On & On & x & x & On & On & On & On & On & x \\  
\hline 
\end{tabular}

\begin{tabular}{|c|c|}
\hline 
Shift Variable & Shift Start Time \\ 
\hline 
s_1 & 7a \\ 
\hline 
s_2 & 3p \\ 
\hline 
\end{tabular}

To specify a valid solution to our tour scheduling problem, we need values for $s_1, s_2$ and $y_1$. For this toy problem, it is easy to see the correspondence between variables in the two approaches. For given values of $x_1$ and $x_2$, $y_1 = x_1 + x_2$, $s_1 = x_1$, and $s_2 = x_2$. While the explicit approach only used two variables, the implicit approach required three variables. However, as we start to model more realistic problems, it will become clear that the implicit approach requires far fewer variables than its explicit counterpart. This reduction will come at a cost in terms of additional constraints as well as in the need for a tour construction post-processing phase. In terms of additional constraints, note that in the implicit approach we must ensure that $y_1 = s_1 + s_2$ if we are going to be able to translate a solution to the implicit problem to an explicit set of tours.

\subsection{Implicit modeling of tours}

Implicit modeling for tour scheduling problems is not new. 

Review one week implicit tour scheduling models including my AOR 2004 paper.

Recent multi-week implicit tour scheduling paper

Also review implicit modeling of breaks and other stuff

\subsection{Problem size explosion}
\label{subsec-explosion}




\section{Related Work}
\label{sec-relatedwork}

If I haven't already covered it above
 

\section{Overview of an Implicit Multi-Week Model}
\label{sec-overview}

As depressingly illustrated in Section \ref{subsec-explosion}, multi-week tour scheduling problems can get massively large. Since even the number of explicit multi-week days worked patterns can get large as the number of weeks increases, our approach is to model these days worked patterns implicitly by focusing on only explicitly modeling the patterns of weekend days worked. This is driven by the observation that it is often the weekends that must be treated in some special way due to the general undesirability of working on weekends as well as the often different required staffing levels seen on weekends. While drastically reducing the total number of variables required, we need to introduce several additional types of variables as well as many control constraints to ensure that a solution to an instance of our implicit model corresponds with a feasible and optimal solution to the equivalent tour scheduling model.



\section{Definitions and notation}
\label{sec-mwts}
%http://en.wikibooks.org/wiki/LaTeX/Mathematics
% Use bold for terms that I define

The notation and definitions presented in this section, are very closely aligned with the implementation of the model in Pyomo. 



Planning cycle related terms...

\subsubsection*{Planning cycle parameters}

\begin{flalign*}
n_W & =  \text{number of weeks in the planning cycle} \\
n_P & =  \text{number of periods per day} \\
n_C & =  7 n_W n_P\mbox{ number of periods in planning cycle} \\
\end{flalign*}

\nipar Define a number of sets of indices.

\begin{flalign*}
\mathbb{P} & =  \{1,2,\ldots n_P\} \\
\mathbb{D} & =  \{1,2,\ldots 7\} \\
\mathbb{W} & =  \{1,2,\ldots n_W\} \\
\end{flalign*}

\nipar So, each period in the planning cycle is defined by a tuple $(i,d,w) \in \mathbb{B}$ where $\mathbb{B} = \mathbb{P} \times \mathbb{D} \times \mathbb{W}$.

 

\subsubsection*{Shift length and tour type parameters}
%#-- Shift Lengths
%model_phase1.n_lengths = Param(within=PositiveIntegers)    # Number of shift lengths
%model_phase1.LENGTHS = RangeSet(1,model_phase1.n_lengths)
%model_phase1.lengths = Param(model_phase1.LENGTHS)  # Vector of shift lengths

\begin{flalign*}
n_K& = \text{number of different shift lengths} \\
\mathbb{K}& = \{1,2,\ldots n_K\} \\
l_k& = \text{$k$'th shift length in periods, for $k \in \mathbb{K}$}
\end{flalign*}




%#-- Tour Types
%model_phase1.n_tts = Param(within=PositiveIntegers)  # Number of different tour types
%model_phase1.TTYPES = RangeSet(1,model_phase1.n_tts)
%model_phase1.tt_length_x = Set(model_phase1.TTYPES,ordered=True,)  # Set of allowable length indices by tour type

\begin{flalign*}
n_T& = \text{number of different tour types} \\
\mathbb{T}& = \{1,2,\ldots n_T\} \\
L(i)& = \text{set of shift length indices allowed for tour type $i$, for $i \in \mathbb{T}$}
\end{flalign*}

A_{ijkt}=\begin{cases}
1& \text{1 if a shift of length $k$ starting in period $i$ of day $j$ for tour type $t$ is allowed},\\
& \text{for $i \in \mathbb{P}$, $j \in \mathbb{D}$, $k \in \mathbb{K}$, $t \in \mathbb{T}$.} \\
0& \text{otherwise}.
\end{cases}

\begin{flalign*}
\mathbb{A} & =  \{(i,j,k,t) | A_{ijkt}=1 \} \\
\end{flalign*}

Instead of explicitly representing tours with one variable per tour, we implicitly represent tours using a number of building block variables whose values are coordinated by numerous control constraints to ensure that valid tours can be constructed from these variables.

\subsubsection*{Shift length and tour type variables}

Shift variables represent a single shift with a start day and time, shift length, in a specific week and corresponding to a certain tour type.

\begin{flalign*}
S_{ijkwt} = & \text{ number of shifts of length $k$ starting in period $i$ of day $j$ in week $w$ for} \\
&  \text{ tour type $t$, for $(i,j,k,t) \in \mathbb(A),t \in T$.}  
\end{flalign*}

\[
S_{ijkwt}& = \mbox{number of shifts of length $k$ starting in period $i$ of day $j$ in week $w$ for \\
tour type $t$, for $(i,j,k,t) \in \mathbb(A),t \in T$}
\]

\begin{flalign*}
T_{it} = & \text{number of tours of tour type $t$ assigned to start time band $i$} \\
&  \text{ tour type $t$, for $(i,j,k,t) \in \mathbb(A),t \in T$.}  
\end{flalign*}
%& = \text(tour type $t$, for $(i,j,k,t) \in \mathbb(A),t \in T$}

%### Bounds on tour type variables
%model_phase1.tt_lb =  Param(model_phase1.TTYPES)       # RHS from .MIX
%model_phase1.tt_ub =  Param(model_phase1.TTYPES,  default=infinity)


























To smooth the postpartum occupancy across days of the week, we introduce two, non-negative, dummy variables, $M^{U}$ and $M^{L}$. These are used to bound the mean occupancy across the week from above and below -- see Equation \eqref{eqn:sandwich}. Then we minimize the difference between the bounds.

\newpage
\nipar {\bf Model:} OB-SMOOTH
\nipar
\nipar Minimize
\begin{gather}
M^{U} - M^{L} \label{eqn:obj} 
\end{gather}
Subject to
\begin{gather}
0 \leq M^{L} \leq \mu_{i,j}^{4} \leq M^{U} \quad \text{($i=1,2,\ldots P$, $j=1,2,\ldots 7$)} \label{eqn:sandwich} \\ 
L_{i,j}^{a} \leq \Lambda_{i,j}^{a} \leq U_{i,j}^{a} \quad \text{($a=5,6, i=1,2,\ldots P$, $j=1,2,\ldots 7$)} \label{eqn:schedbin} \\
L_{.,j}^{a} \leq \sum_{i=1}^{P}\Lambda_{i,j}^{a} \leq U_{.,j}^{a} \quad \text{($a=5,6$, $j=1,2,\ldots 7$)} \label{eqn:schedday} \\
L^{a} \leq \sum_{i=1}^{P}\sum_{j=1}^{7}\Lambda_{i,j}^{a} \leq U^{a} \quad \text{($a=5,6$)} \label{eqn:schedweek} \\ 
\text{Equations \eqref{eqn:arrrate1}-\eqref{eqn:arrrate11} for arrival rates }  \\
\text{Equation \eqref{eqn:discharge} for discharge rates}  \\
\text{Equation \eqref{eqn:flowcons} for conservation of flow}  \\
\text{Equation \eqref{eqn:meanocc} for mean occupancy} \\
\text{Equations \eqref{eqn:varocc2}-\eqref{eqn:varocc1} for variance of occupancy }  \\
\Lambda_{i,j}^{a} \geq 0 \text{ and integer}, \quad \text{($a=5,6, i=1,2,\ldots P$, $j=1,2,\ldots 7)$} \label{eqn:schedint}
\end{gather} 

Equation \eqref{eqn:schedbin} sets upper and lower bounds on the number of scheduled inductions and scheduled cesareans by time period of day and day of week. Equations \eqref{eqn:schedday} and \eqref{eqn:schedweek} set upper and lower bounds on the number of scheduled inductions and scheduled cesareans by day of week and for the entire week, respectively. The remaining constraints constitute the patient flow portion of the model. Note that the only decision variables are $\Lambda_{i,j}^{5}$, the number of scheduled inductions, and $\Lambda_{i,j}^{6}$, the number of scheduled cesareans. The model OB-SMOOTH is a standard mixed integer linear program. 


\subsection{Computational experiments}
\label{sec-computational}

Solve realistic problems. No competitive models to which to compare to other than reporting the size of comparable explicit tour scheduling models. I guess I could compare to the 5/7 model for the restricted environment for which it was designed.

\section{Just some templates to use}


\begin{table}
  \centering
  \caption{Patient care units}\label{table:units}
\begin{tabular}{cll}\hline
 Unit \#       & Unit Name    & Abbr.   \\ \hline
  1       & Labor \& Delivery    & LD    \\
  2       & Recovery room   & R    \\
  3       & Cesarean section procedure area    & C    \\
  4       & Postpartum unit    & PP   \\ \hline  
\end{tabular}
\end{table}



\begin{equation}
\label{eqn:flowcons}
\lambda_{i,j}^{h,s(h,m)}=D_{i,j}^{h,r(h,m)} \quad \text{($i=1,2,\ldots P$, $j=1,2,\ldots 7$, $h \in {\cal H}$, $m=2,3\ldots n_{h}$).}
\end{equation}
\section{Conclusions and Future Work}
\label{sec-conclusions}


BLAH, BLAH






%\begin{acknowledgements}
%If you'd like to thank anyone, place your comments here
%and remove the percent signs.
%\end{acknowledgements}


% BibTeX users please use one of
% BibTeX users please use one of
%\bibliographystyle{spbasic}      % basic style, author-year citations
%\bibliographystyle{spmpsci}      % mathematics and physical sciences
%\bibliographystyle{spphys}       % APS-like style for physics
%\bibliography{obsched-paper}   % name your BibTeX data base
%\bibliographystyle{unsrt}


\end{document}
% end of file template.tex

