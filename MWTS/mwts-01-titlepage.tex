%%%%%%%%%%%%%%%%%%%%%%% file template.tex %%%%%%%%%%%%%%%%%%%%%%%%%
%
% This is a general template file for the LaTeX package SVJour3
% for Springer journals.          Springer Heidelberg 2006/03/15
%
% Copy it to a new file with a new name and use it as the basis
% for your article. Delete % signs as needed.
%
% This template includes a few options for different layouts and
% content for various journals. Please consult a previous issue of
% your journal as needed.
%
%%%%%%%%%%%%%%%%%%%%%%%%%%%%%%%%%%%%%%%%%%%%%%%%%%%%%%%%%%%%%%%%%%%
%
% First comes an example EPS file -- just ignore it and
% proceed on the \documentclass line
% your LaTeX will extract the file if required
\begin{filecontents*}{example.eps}
%!PS-Adobe-3.0 EPSF-3.0
%%BoundingBox: 19 19 221 221
%%CreationDate: Mon Sep 29 1997
%%Creator: programmed by hand (JK)
%%EndComments
gsave
newpath
  20 20 moveto
  20 220 lineto
  220 220 lineto
  220 20 lineto
closepath
2 setlinewidth
gsave
  .4 setgray fill
grestore
stroke
grestore
\end{filecontents*}
%
\documentclass{svjour3}                     % onecolumn (standard format)
%\documentclass[smallextended]{svjour3}     % onecolumn (second format)
%\documentclass[twocolumn]{svjour3}         % twocolumn
%
\smartqed  % flush right qed marks, e.g. at end of proof
%
\usepackage{graphicx}
%
\usepackage{mathptmx}      % use Times fonts if available on your TeX system
%
% insert here the call for the packages your document requires
%\usepackage{latexsym}
% etc.
\usepackage{amsmath, amssymb}
%\usepackage{amsthm}
%
% please place your own definitions here and don't use \def but
\newcommand{\nipar}{\par\noindent\ignorespaces}

%
% Insert the name of "your journal" with
\journalname{Health Care Management Science}
%
\begin{document}

\title{An open source software project for obstetrical procedure scheduling and occupancy analysis%\thanks{Grants or other notes
%about the article that should go on the front page should be
%placed here. General acknowledgments should be placed at the end of the article.}
}
%\subtitle{Do you have a subtitle?\\ If so, write it here}

%\titlerunning{Short form of title}        % if too long for running head

\author{Mark W. Isken         \and
        Timothy J. Ward         \and
        Steven J. Littig         \and
}

%\authorrunning{Short form of author list} % if too long for running head

\institute{M. Isken \at
              School of Business Administration, Oakland University, Rochester, MI \\
              Tel.: 248-370-3296\\
              Fax: 248-370-4275
              \email{isken@oakland.edu}           
           \and
           T. Ward \at
              Bureau of Medicine and Surgery, United States Navy, Washington, DC
 \and
           S. Littig \at
              Improvement Path Systems, Inc., Farmington Hills, MI
}

%\date{Received: date / Accepted: date}
% The correct dates will be entered by the editor


\maketitle

\begin{abstract}
Increases in the rate of births via cesarean section and induced labor have led to some challenging scheduling and capacity planning problems for hospital inpatient obstetrical units. We present occupancy and patient scheduling models to help address these challenges. We make heavy use of open source software tools and have released entire project as a free and open source model and software toolkit.
\keywords{patient flow \and scheduling \and obstetrics \and open source software}
% \PACS{PACS code1 \and PACS code2 \and more}
% \subclass{MSC code1 \and MSC code2 \and more}
\end{abstract}


\end{document}
% end of file template.tex

